\documentclass{beamer}

\mode<presentation>
\usetheme{Madrid}

\usepackage[spanish]{babel}
\usepackage[utf8]{inputenc}
\usepackage[T1]{fontenc} % hyphenation 
\usepackage{times}
\usepackage{tikz}

\setbeamercovered{dynamic}

\title[GIS]{Introducción a Sistemas de Información Geográfica}
\subtitle[]{Curso de Zonificación Vitícola y Viticultura de Precisión}


\author[G.F. Olmedo]{Ing. Agr. Guillermo Federico Olmedo}

\institute[INTA] % (optional, but mostly needed)
{ Laboratorio de Geomática\\
  Recursos Naturales\\
  INTA EEA Mendoza\\
  \vskip10pt
\begin{columns}
	\column{.5\textwidth}
	\begin{flushright}
		\includegraphics[width=1.7cm]{logoINTA}
	\end{flushright}
	\column{.5\textwidth}
	\begin{flushleft}
		\includegraphics[width=1.7cm]{yo}
	\end{flushleft}
\end{columns}  
}

\date[FCA, 16-20/05/2016]{Fac. de Cs. Agrarias, 16 al 20 de mayo de 2016}

\pgfdeclareimage[height=0.5cm]{university-logo}{logoINTA}
\logo{\pgfuseimage{university-logo}}

%\beamerdefaultoverlayspecification{<+->}


\begin{document}

\begin{frame}[plain]
  \titlepage
\end{frame}

\begin{frame}{Outline}
	\tableofcontents[pausesections]
	% You might wish to add the option [pausesections]
\end{frame}

\section{Introducción}

\begin{frame}{Que es un SIG?}
Un Sistema de Información Geográfica (SIG o GIS)\footnote{{\footnotesize }Fuente: Wikipedia}, es una integración organizada de \textit{hardware}, \textit{software} y datos \textit{geográficos} diseñada para capturar, almacenar, manipular y analizar información \textit{geográficamente referenciada} con el fin de resolver problemas complejos de planificación y gestión.
\end{frame}

\begin{frame}[plain]{}
	\includegraphics[width=\textwidth]{googleea}
\end{frame}

\begin{frame}[plain]{}
	\includegraphics[width=\textwidth]{QGIS}
\end{frame}

\begin{frame}[plain]{}
	\includegraphics[width=\textwidth]{GIS}
\end{frame}

\section{Información}

\begin{frame}{Tipos de información espacial}{Raster}
	\begin{columns}
		\column{.45\textwidth}
			\begin{itemize}
				\item Es una matriz ordenada de celdas en filas y columnas. Cada celda (píxel) toma un valor único. Generalmente representan datos continuos, son más importantes las propiedades del espacio que la precisión de la localización.
				\item<6> En algunos formatos se suelen “apilar” diferentes variables, pudiendo luego ver el valor de las variables para una misma ubicación.
			\end{itemize}
		\column{.45\textwidth}
		    \begin{figure}
		    	\includegraphics<1>[width=0.8\textwidth]{raster1}
		    	\includegraphics<2>[width=0.8\textwidth]{raster2}
		    	\includegraphics<3>[width=0.8\textwidth]{raster3}
		    	\includegraphics<4>[width=0.8\textwidth]{raster4}
		    	\includegraphics<5>[width=0.8\textwidth]{raster5}
		    	\includegraphics<6>[width=0.9\textwidth]{multi}
		    \end{figure} 
	\end{columns}
\end{frame}

\begin{frame}{Tipos de información espacial}{Tipos de Raster}
	\begin{itemize}[<+->]
		\item Raster continuos\\
		Generalmente representan una propiedad física continua. 
		Por ejemplo: la altitud, la reflectancia a una longitud de onda determinada o el valor de NDVI.
		\item Raster temáticos\\
		Los píxels toman un valor discreto que corresponde a una clase. Una variación de estos son los raster booleanos, que toman valor 1 o 0 de acuerdo a la pertenencia o no a una clase.
		Por ej.: zonas de alto o bajo vigor; suelos con freática o no.
	\end{itemize}
\end{frame}

\begin{frame}{Tipos de información espacial}{Vectorial}
	\begin{columns}
		\column{.45\textwidth}
		\begin{itemize}
			\item Pueden ser puntos, líneas o polígonos. 
			\item Generalmente representan datos discretos, donde la precisión de la localización en el factor más importante. 
			\item Cada entidad esta ligada a una base de datos.
		\end{itemize}
		\column{.45\textwidth}
		\begin{figure}
			\includegraphics[width=0.8\textwidth]{vector}
		\end{figure}
	\end{columns}
\end{frame}

\begin{frame}{title}
	\begin{columns}[t]
		\column{.45\textwidth}
		Raster
		\begin{itemize}
			\item Información continua
			\item Análisis entre capas
		\end{itemize}
		\column{.45\textwidth}
		Vectorial
		\begin{itemize}
			\item Información discreta
			\item Menos volumen de datos
			\item Representación independiente de la escala
		\end{itemize}
	\end{columns}
\end{frame}



\end{document}

